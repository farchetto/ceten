%%%%%%%%%%%%%%%%%%%%%%%%%%%%%%%%%%%%%%%%%%%%%%%%%%%%%%%%%%%%%%%%%%%%%%%%%%%%%%%%
%                                                                              %
%                             Statuts du Ceten                                 %
%                                                                              %
%%%%%%%%%%%%%%%%%%%%%%%%%%%%%%%%%%%%%%%%%%%%%%%%%%%%%%%%%%%%%%%%%%%%%%%%%%%%%%%%
%
%  Version LaTeX des statuts du Ceten tels que remaniés lors de l'assemblée
%  générale du lundi 14 novembre 2016.
%
%  Licence :
%    CC-BY-SA 4.0 (http://creativecommons.org/licenses/by-sa/4.0)
%
%%%

%%%%%%%%%%%%%%%%%%%
%  Configuration  %
%%%%%%%%%%%%%%%%%%%

\documentclass{article}
\usepackage[a4paper,includeheadfoot,margin=2.54cm]{geometry}
\usepackage[francais]{babel}

\usepackage[bookmarks=false]{hyperref}
\usepackage{graphicx} 
\usepackage{titlesec}
\usepackage{changepage}
\usepackage[usenames]{xcolor} 
\definecolor{newCeten}{RGB}{130,11,95}
\usepackage{multicol}
\widowpenalties1 10000
\raggedbottom{}

\usepackage{fontspec} 
\setmainfont{Roboto Slab}
\setsansfont{Roboto}
\newfontfamily\condensed{Roboto Condensed}
\newfontfamily\condensedlight{Roboto Condensed Light}
\newfontfamily\light{Roboto Slab Light}
\titleformat*{\section}{\LARGE\condensed\color{newCeten}}
\titleformat*{\subsection}{\Large\condensedlight\color{newCeten}}
\titleformat*{\subsubsection}{\large\condensedlight\color{newCeten}}

\title{Statuts du Ceten}
\author{Philippe Becker\\
	Caroline Boyer\\
	Kilian Cuny\\
	Julien Déoux\\
	Yoni Lévy} % Ajoutez vous. Plus on est de fous, plus on rit.
\date\today

%%%%%%%%%%%%%%
%  Document  %
%%%%%%%%%%%%%%

\begin{document}

	\pagenumbering{roman}

	%---------------%
	% Page de garde %
	%---------------%
	
	\begin{titlepage}
		\begin{center}
			\includegraphics[width=\textwidth]{images/ceten.png}\par
			\vspace{3cm}
			{\Huge\light{} Statuts}\par
			\vfill
			{\large Cercle des Élèves de TELECOM Nancy}\par
			{\large\light{} Association déclarée}
			\vfill
			{\light{} 14 novembre 2016}\par
		\end{center}
	\end{titlepage}

	%----------%
	% Articles %
	%----------%

	\pagenumbering{arabic}

	\section{Constitution et dénomination}
\label{sec:consitution_et_denomination}
		Il est fondé, depuis le 10 janvier 1991 entre les adhérents aux présents
		statuts, une association à but non lucratif régie par la loi du 1\up{er}
		Juillet 1901 et le décret du 16 août 1901, ayant pour titre “Cercle des
		Élèves de TELECOM Nancy” et pour sigle “CETEN”.

	\section{Objet et buts}
\label{sec:objet_et_buts}
		Cette association a pour objet d’aider les élèves adhérents, anciens
		comme nouveaux, de TELECOM Nancy. Pour cela, elle se donne comme
		principaux objectifs:
		\begin{itemize}
			\item L’aide à la gestion de la vie extrascolaire en collaboration
				avec les autres associations d'élèves de TELECOM Nancy, en
				proposant des services et activités à l’ensemble de ses membres
				adhérents
			\item De favoriser l’entraide et la solidarité entre ses membres
			\item L’insertion des nouveaux élèves en première année
			\item De défendre et représenter les intérêts de ses adhérents
			\item Tout autre objectif non-spécifique permettant la réalisation
				des objectifs susmentionnés.
		\end{itemize}

	\section{Siège social}
\label{sec:siege_social}
		Le siège social est fixé:
		\begin{center}
			TELECOM Nancy\\
			193, Avenue Paul Muller\\
			CS 90172\\
			54602 Villers-lès-Nancy, FRANCE
		\end{center}

		Le transfert du siège social pourra être effectué par décision de
		l’assemblée générale.

	\section{Les caractères}
\label{sec:les_caracteres}
		L’association prône des valeurs apolitiques, asyndicales et
		aconfessionnelles. Elle est ouverte à tout élève de TELECOM Nancy, ainsi
		qu’à toute personne dont la demande d’adhésion a été acceptée par le
		bureau des élèves, sans distinction de race, d'âge, de sexe, de
		nationalité, de religion, d’opinion politique, d’orientation sexuelle,
		de condition physique.

	\section{Durée de l’association}
\label{sec:duree_de_l_association}
		La durée de l’association est illimitée.

	\section{Composition de l'association}
\label{sec:composition_de_l_association}
	
		L’association se compose de:
		\begin{itemize}
			\item Membres de fait
			\item Membres bienfaiteurs
			\item Membres adhérents
			\item Membres ponctuels % À discuter dans les remaniements futurs
				% Utilité alors que les membres extérieurs existent ?
			\item Membres extérieurs.
		\end{itemize}
	
		Toutes ces personnes sont des personnes physiques.

	\section{Les membres}
\label{sec:les_membres}
	
		\subsection{Admission et adhésion}
\label{sub:admission_et_adhesion}
		
			Pour faire partie de l'association, il faut:
			\begin{itemize}
				\item Dans le cas des membres adhérents, être ou avoir été élève
					de TELECOM Nancy
				\item Adhérer aux présents statuts
				\item À l’exception des membres de fait, bienfaiteurs et
					ponctuels, s'acquitter de la cotisation dont le montant est
					fixé par le bureau des élèves.
			\end{itemize}

			Le bureau des élèves pourra refuser des adhésions, avec avis motivé
			aux intéressés.

		\subsection{Les catégories de membres}
\label{sub:les_categories_de_membres}
			\subsubsection{Les membres de fait}
\label{ssub:les_membres_de_fait}
				Peut être membre de fait:
				\begin{itemize}
					\item Tout ancien membre adhérent
					\item Tout membre actif d’une association étudiante en
						rapport avec TELECOM Nancy
					\item Tout enseignant, intervenant ou personnel de TELECOM
						Nancy
				\end{itemize}
				qui en fait la demande sous réserve d’acceptation du bureau des
				élèves.

				Ce titre est décerné pour l’année scolaire en cours (de la date
				d’adhésion jusqu’au 31 août suivant) par le bureau des élèves à
				l’exception du directeur de TELECOM Nancy qui est membre de fait
				permanent. Les membres de fait ont droit de participation et de
				vote aux assemblées générales.

			\subsubsection{Les membres bienfaiteurs}
\label{ssub:les_membres_bienfaiteurs}
				Peut être membre bienfaiteur toute personne physique ayant fait
				un don manuel à l’association sous réserve d’acceptation du
				bureau des élèves.

			\subsubsection{Les membres ponctuels}
\label{ssub:les_membres_ponctuels}
			
				Peut être membre ponctuel toute personne physique souhaitant
				participer à une manifestation ponctuelle ayant payé une
				cotisation d’un montant défini par le bureau des élèves pour
				ladite manifestation, sous réserve d’acceptation du bureau des
				élèves. Il ou elle n’est membre que pendant la durée de ladite
				manifestation.

			\subsubsection{Les membres adhérents}
\label{ssub:les_membres_adherents}
			
				Est membre adhérent, toute personne souhaitant profiter des
				services et activités proposés par l’association qui s’est
				acquittée de la cotisation fixée annuellement. Les membres
				adhérents ont droit de participation et de vote aux assemblées
				générales.

			\subsubsection{Les membres extérieurs}
\label{ssub:les_membres_exterieurs}
			
			% Ajoutés lors du mandat 2016, les membres extérieurs ont fait
			% débat pour différentes raisons (équité, compatibilité avec l'objet
			% de l'association...). Si vous lisez ceci parce que vous avez
			% l'intention de supprimer cette catégorie... bah désolés...
				Peut être membre extérieur toute personne non étudiante ou
				alumni de TELECOM Nancy souhaitant rejoindre l’association et
				s'étant acquittée d’une cotisation annuelle réduite fixée par le
				bureau des élèves, sous réserve d'acceptation du bureau des
				élèves. Cet acquittement n’est valide que pour l’année scolaire
				en cours (de la date d’adhésion jusqu’au 31 août suivant).

				Les membres extérieurs peuvent uniquement prétendre aux tarifs
				CETEN lors des manifestations organisées par l’association ainsi
				qu’aux avantages liés aux partenariats conclus avec
				l’association. Ils n’ont pas accès aux autres services ou
				activités de l’association. Un membre extérieur n’a pas le droit
				de participation ni de vote aux assemblées générales.

		\subsection{Perte de qualité de membre}
\label{sub:perte_de_qualite_de_membre}
			La qualité de membre se perd par:
			\begin{itemize}
				\item Décès
				\item Radiation prononcée par le bureau des élèves pour
					non-paiement de la cotisation, l'intéressé ayant été invité
					à fournir des explications devant ledit bureau
				\item Exclusion prononcée par le bureau des élèves pour
					infraction aux présents statuts, au règlement intérieur de
					l'association, ou pour tout autre motif portant préjudice
					aux intérêts moraux et matériels de l’association,
					l'intéressé ayant été invité à fournir des explications
					devant ledit bureau et par écrit
				\item Démission adressée par écrit au président de l’association
			\end{itemize}

			Un membre ayant été exclu ne pourra prétendre à une nouvelle
			adhésion qu’après avis favorable du bureau et en tout état de cause,
			pas avant l’année scolaire suivant son exclusion.

	\section{Affiliation}
\label{sec:affiliation}
		La présente association peut adhérer à d’autres associations, unions ou
		regroupements par décision du bureau des élèves tant que cette
		affiliation n’entre pas en conflit avec le caractère et les objectifs de
		l’association décrits dans les articles~\ref{sec:objet_et_buts}
		et~\ref{sec:les_caracteres} des présents statuts.

	\section{Ressources et financement}
\label{sec:ressources_et_financement}
		\subsection{Ressources}
\label{sub:ressources}
			Les ressources de l’association comprennent:
			\begin{itemize}
				\item Le montant des cotisations
				\item Les dons manuels
				\item Les subventions diverses
				\item Les manifestations
				\item Toutes les ressources autorisées par les lois et le
					règlement en vigueur
			\end{itemize}

		\subsection{Exercice comptable}
\label{sub:exercice_comptable}
			L’exercice comptable commence au 1\up{er} janvier et finit le 31
			décembre de chaque année. Tout membre de l’association peut, sur
			simple demande au trésorier et en présence de celui-ci, accéder aux
			comptes de l’association.

	\section{Organes souverains}
\label{sec:organes_souverains}
		L’association est administrée par:
		\begin{itemize}
			\item L’assemblée générale
			\item Le bureau des élèves
		\end{itemize}
		% On remarquera l'absence de majuscules dans le nom des organes du
		% Ceten. Plus d'infos ici :
		% http://larevue.squirepattonboggs.com/Post-it-Du-bon-emploi-de-la-
		% majuscule-par-les-juristes_a246.html

		\subsection{Dispositions communes aux deux assemblées générales}
\label{sub:dispositions_communes_aux_deux_assemblees_generales}
		
			\subsubsection{Composition des assemblées générales}
\label{ssub:composition_des_assemblees_generales}
			
				Les assemblées générales, ordinaire ou extraordinaire, peuvent
				comprendre tous les membres de l’association ayant explicitement
				droit de participation aux assemblées générales d'après la
				Section~\ref{sub:les_categories_de_membres}.

			\subsubsection{Convocation et ordre du jour}
\label{ssub:convocation_et_ordre_du_jour}
				Quinze jours avant la date fixée par le bureau des élèves, les
				membres de l’association sont convoqués par les soins du
				secrétaire de l’association. La convocation se fait par courrier
				électronique. L’ordre du jour fixé et prévu par le bureau des
				élèves, sera envoyé aux membres en même temps que la convocation
				et sera modifiable jusqu’à sept jours avant l'assemblée
				générale. Seuls les points inscrits à l’ordre du jour pourront
				être abordés.
				
				Tout membre de l'association peut, s'il le souhaite, soumettre
				un point à ajouter à l'ordre du jour au bureau des élèves sous
				condition que cet ajout soit constructif et non contraire au but
				et au caractère de l'association. Ce membre, ou le membre qui le
				représente, aura l'obligation dans le cas ou son point est
				ajouté par validation du bureau des élèves, de faire entendre à
				l'assemblée générale son avis sur ledit point sans quoi celui-ci
				ne sera pas abordé.

			\subsubsection{Président et secrétaire de l'assemblée générale}
\label{ssub:president_et_secretaire_de_l_assemblee_generale}
			
				La présidence de l’assemblée générale appartient au président de
				l’association ou à un membre du bureau des élèves choisi par
				celui-ci s'il est empêché.

				La rédaction du procès verbal de l’assemblée générale appartient
				au secrétaire de l’association ou à un membre du bureau des
				élèves s’il est empêché. Les délibérations sont constatées par
				des procès-verbaux inscrits sur un registre et signés par le
				président et le secrétaire de séance.

			\subsubsection{Délibérations}
\label{ssub:deliberations}
			
				Les délibérations se font à main levée sauf demande de la part
				d’un membre adhérent. Les décisions sont prises à la majorité
				des suffrages exprimés, c’est-à-dire des membres présents ou
				représentés. En cas de partage, la voix du président est
				prépondérante.

			\subsubsection{Représentation des membres en cas d'empêchement}
\label{ssub:representation_des_membres_en_cas_d_empechement}

				Les membres de l’association peuvent se faire représenter par un
				autre membre de l’association en cas d’empêchement. Un membre
				présent ne peut détenir plus d’un mandat de représentation.

				Doit être tenue une feuille de présence signée par chaque membre
				présent et certifiée par le président de l’assemblée.

		\subsection{L'assemblée générale ordinaire (AGO)}
\label{sub:l_assemblee_generale_ordinaire_ago_}
		
			Elle se réunit chaque année au mois de janvier. Elle peut délibérer
			valablement quel que soit le nombre de membres présents ou
			représentés. Le président de l’assemblée générale fait entendre à
			l’assemblée générale la situation morale et l’activité de
			l’association. Le trésorier de l’association rend compte à
			l’assemblée générale de la gestion financière. Après avoir délibéré
			et statué les divers rapports, l’assemblée générale délibère sur le
			budget prévisionnel ainsi que sur les autres points à l’ordre du
			jour.

		\subsection{L'assemblée générale extraordinaire (AGE)}
\label{sub:l_assemblee_generale_extraordinaire_age_}
		
			Elle peut se réunir à la demande du bureau des élèves ou à la
			demande de la moitié plus un des membres de l’association, sur
			convocation du président de l’association, suivant les modalités
			prévues dans l’article~\ref{ssub:convocation_et_ordre_du_jour}.

			\subsubsection{Quorum}
\label{ssub:quorum}
		
				Dans le cas de l’assemblée générale extraordinaire, les
				délibérations ne peuvent être validées que si au moins le tiers
				des membres de l’association est représenté. Si le quorum n’est
				pas atteint, une nouvelle assemblée générale extraordinaire est
				convoquée dans un délai d’au moins 7 jours, avec le même ordre
				du jour mais sans nécessité de quorum.

				Un membre représentant le quorum de l’assemblée générale peut,
				s’il le souhaite, soumettre au président de l’association un
				point à ajouter à l’ordre du jour. Ce point sera ajouté s’il a
				été soumis au moins 3 jours avant l’assemblée générale
				extraordinaire, sous condition que cet ajout soit constructif et
				non contraire à l'objet et au caractère de l’association.

			\subsubsection{Modification des statuts}
\label{ssub:modification_des_statuts}
			
				Les statuts de l’association peuvent être modifiés à la demande
				de la majorité du bureau des élèves. Tout membre peut, s’il le
				souhaite, soumettre une idée de modification des statuts au
				bureau des élèves lors des réunions dudit bureau, sous condition
				que cette modification soit constructive et non contraire au but
				et au caractère de l’association, sauf si elle concerne
				l'article~\ref{sec:objet_et_buts} ou
				l'article~\ref{sec:les_caracteres}.

				Toute modification des statuts doit être rédigée et envoyée aux
				membres de l’association au plus tard en même temps que la
				convocation pour l’assemblée générale extraordinaire durant
				laquelle seront votés les changements de statuts.

			\subsubsection{Dissolution}
\label{ssub:dissolution}
			
				La dissolution de l’association ne peut être prononcée que par
				l’assemblée générale extraordinaire convoquée spécialement à cet
				effet. En cas de dissolution de l’association, l’actif net
				subsistant et les documents comptables seront remis
				obligatoirement à une ou plusieurs associations poursuivant des
				buts similaires et qui seront désignées par l’assemblée générale
				extraordinaire. Les comptes afférents seront clos. La
				dissolution doit faire l’objet d’une déclaration à la préfecture
				ou à la sous-préfecture du siège social.

		\subsection{Le bureau des élèves}
\label{sub:le_bureau_des_eleves}
			\subsubsection{Fonction}
\label{ssub:fonction}
			
				Le bureau des élèves (BDE) est l'organe exécutif de
				l'association, il est placé sous la direction du président. Il
				assure la gestion de l'association entre deux assemblées
				générales dans le but de mettre en oeuvre les décisions de la
				dernière assemblée générale et conformément à l'objet des
				statuts.

			\subsubsection{Composition}
\label{ssub:composition}
			
				Le bureau des élèves est composé de 10 élèves de TELECOM Nancy
				élus pour un mandat d’un an, débutant le 1\up{er} janvier et
				finissant le 31 décembre. Dans les deux semaines suivant son
				élection, le nouveau bureau des élèves doit se réunir afin
				d'attribuer les postes suivants:
				\begin{itemize}
					\item Un Président, élève de deuxième année
					\item Un Vice-président, élève de première année
					\item Un Trésorier assisté d’un Vice-trésorier
					\item Un Secrétaire
					\item Un Responsable des clubs
					\item Un Responsable informatique
					\item Un Responsable logistique
					\item Deux Responsables communication.
				\end{itemize}

				Le cumul des postes énumérés ci-dessus est interdit.

				En plus de ces 10 élèves:
				\begin{itemize}
					\item est choisi en la personne du président sortant un
						représentant des troisième année. Dans le cas où le
						président sortant n’est pas en troisième année à TELECOM
						Nancy au moment des élections ou est dans l'incapacité
						d'assurer cette fonction, un autre membre du bureau sera
						alors choisi par le bureau des élèves sortant. Dans le
						cas extrême où aucun membre du bureau sortant n’est en
						troisième année, le bureau sortant choisira un membre
						adhérent, élève de troisième année.
					\item est élu par l’ensemble des membres adhérents en
						filière apprentissage un responsable apprentis.
				\end{itemize}
				% Piste de réflexion pour l'avenir : pourquoi ne pas simplement
				% statuer que le BDE peut nommer des membres supplémentaires
				% n'ayant pas le droit de vote et définis dans le RI ? Cela
				% permettrait plus de souplesse pour des postes qui de toute
				% façon ne sont pas déclarés dans le procès verbal de l'AGO.

			\subsubsection{Délibérations}
\label{ssub:deliberations}
			
				Les délibérations se font à main levée sauf demande de la part
				d’un membre du bureau des élèves. Les décisions sont prises à la
				majorité des suffrages exprimés, c’est-à-dire des membres
				présents ou représentés. En cas d'égalité, la voix du président
				est prépondérante.

				L’ensemble de ses pouvoirs doit être soumis à la validation de
				la majorité absolue du bureau des élèves. Les membres du bureau
				des élèves peuvent se faire représenter par un autre membre du
				bureau en cas d’empêchement. Un membre présent ne peut détenir
				plus d’un mandat de représentation.

			\subsubsection{Pouvoirs}
\label{ssub:pouvoirs}
			
				Le bureau des élèves:
				\begin{itemize}
					\item Peut autoriser tout acte ou opération qui n’est pas
					    statutairement de la compétence de l’assemblée générale
					    ordinaire ou extraordinaire.
					\item Se prononce sur les admissions de membres de
						l’association et confère les éventuels titres de membres
						de fait et bienfaiteurs.
					\item Se prononce également sur les mesures de radiation et
	    				d’exclusion des membres.
					% \item Contrôle la gestion des clubs qui doivent lui rendre
					%   compte de leur activité à l’occasion de ses
					%   réunions.
						% Ce point a été retiré lors du changement des statuts
						% de 2016 pour supprimer toute référence aux clubs des
						% statuts. La gestion des clubs est désormais
						% entièrement régie par le règlement intérieur et la
						% charte des clubs car elle relève du fonctionnement
						% interne de l'association
					\item Autorise l’ouverture de tous comptes bancaires ou
						postaux, effectue tout emploi de fonds, contracte tout
						emprunt hypothécaire ou autre, sollicite toute
						subvention, requiert toutes inscriptions ou
						transcriptions utiles.
					\item Autorise le président ou le trésorier à exécuter tout
						acte, aliénation ou investissement reconnu nécessaire,
						des biens et des valeurs appartenant à l’association et
						à passer les marchés et contrats nécessaires à la
						poursuite de son objet.
				\end{itemize}

			\subsubsection{Modalités des élections du bureau des élèves}
\label{ssub:modalites_des_elections_du_bureau_des_eleves}
			
				L’élection se fait exclusivement à bulletin secret. Le vote par
				procuration est admis dans la limite d’une procuration par
				personne. Elle doit être datée et signée, rédigée sur les
				formulaires fournis par le bureau des élèves. Le vote par
				correspondance est interdit. Dans le cas où l’électeur est
				porteur d’une procuration, il doit introduire autant de
				bulletins que le nombre de voix qu’il représente et doit émarger
				devant le nom de la personne dont il porte la procuration. Il
				doit remettre ses procurations au responsable du bureau de vote.
				
				Dans le cas où il y a une ou deux listes, seul un tour est
				nécessaire. Pour que les bulletins soient déclarés recevables,
				on effectue alors un scrutin plurinominal majoritaire à un tour
				avec panachage dont le détail du fonctionnement est le suivant:

				\begin{adjustwidth}{1cm}{}
					Sont élus les cinq premiers candidats de première année et
					les cinq premiers candidats de deuxième année ayant eu le
					plus de voix à condition que le nombre total de votants soit
					supérieur à 33\% des électeurs inscrits.

					Dans le cas où le nombre de votants n’est pas supérieur à
					33\% des électeurs inscrits, le vote est prolongé d’un jour
					ouvré. Si à la fin de ce jour, le nombre de votants n’est
					toujours pas supérieur à 33\% des électeurs inscrits, le
					dépouillement a quand même lieu.
				\end{adjustwidth}

				Dans le cas où il y a plus de deux listes, pour que les
				bulletins soient déclarés recevables, on effectue alors un
				scrutin plurinominal majoritaire à deux tours avec panachage
				dont le détail du fonctionnement est le suivant:

				\begin{adjustwidth}{1cm}{}
				Sont élus au premier tour dans l’ordre décroissant des voix les
					candidats ayant eu une majorité absolue des voix à condition
					que le nombre de voix recueillies soit supérieur à 33\% des
					électeurs inscrits. Les votes pour et blanc comptent dans le
					nombre total de voix exprimées. Les votes nuls ne comptent
					pas dans le nombre total de voix exprimées.

					Si cinq étudiants de première année et cinq de deuxième
					année sont élus, les élections s’achèvent.
				
					Dans le cas contraire, un second tour est organisé dans un
					délai de cinq jours ouvrables. Par année d’étude, seuls
					peuvent se maintenir les candidats non élus ayant obtenu les
					voix d’au moins 25\% des inscrits ou, à défaut, les six
					candidats non élus ayant obtenu le plus de voix. En cas
					d’égalité du nombre de voix au premier tour et si cette
					égalité empêche de choisir les candidats admissibles au
					second tour, ces candidats sont admis au second tour.

					Au second tour, en cas d'égalité de suffrages entre
					plusieurs élèves d'une même année et si cette égalité
					empêche de désigner les cinq élèves entrant au Bureau des
					Élèves, un nouveau tour portant sur ces seuls élèves sera
					organisé dans un délai de cinq jours ouvrables afin de les
					départager, et ainsi de suite jusqu’à départage entre ces
					candidats.
				\end{adjustwidth}

				Dans tous les cas, tout bulletin incomplet, corrigé, raturé ou
				comprenant une marque distinctive sera déclaré nul. Tout
				bulletin sans aucune écriture sera considéré comme blanc.

			\subsubsection{Vacances}
\label{ssub:vacances}
			
				En cas de vacance d'un poste, le bureau des élèves pourvoit
				provisoirement au remplacement du membre. Il sera procédé à son
				remplacement définitif lors de la prochaine assemblée générale.
				Les pouvoirs des membres ainsi élus prennent fin à l’époque où
				devrait normalement expirer le mandat des membres remplacés. Si
				plus d’un tiers des membres du bureau des élèves devaient
				laisser leur poste vacant, la nouvelle composition du bureau
				devra être validée par une assemblée générale extraordinaire.

				La vacance du poste de Président ou de Trésorier entraînera
				obligatoirement la convocation d’une assemblée générale
				extraordinaire dans un délai d’un mois afin de prendre les
				mesures qui s’imposent.

	\section{Règlement intérieur}
\label{sec:reglement_interieur}
	
		Un règlement intérieur sera déposé par le bureau des élèves et adopté
		par ledit bureau à la majorité absolue afin de fixer les divers points
		non prévus dans les statuts, notamment ceux qui ont trait au descriptif
		des postes du bureau, au mode d’utilisation des locaux et des divers
		équipements, à l’administration interne de l’association et aux montants
		des cotisations. Il ne pourra comprendre aucune disposition contraire
		aux statuts.

		Ce règlement intérieur peut être modifié par le bureau des élèves. Il
		est mis à disposition à l’ensemble des membres ainsi qu’à chaque nouvel
		adhérent. Le nouveau règlement intérieur sera adressé à chacun des
		membres de l'association par affichage et courrier électronique sous un
		délai de 7 jours suivant la date de la modification. Le délai
		d’application est de 7 jours après la diffusion du règlement intérieur.

		Les articles concernant l’utilisation des locaux seront déposés par le
		bureau des élèves et adoptés en accord avec le directeur de TELECOM
		Nancy. Leur modification nécessite l’accord du directeur de TELECOM
		Nancy.

	\vspace*{5cm}
	\begin{center}
		{\large\light{} Les présents statuts ont été votés lors de l’assemblée
		générale extraordinaire à Villers-lès-Nancy le lundi 14 novembre 2016.}
	\end{center}
    \vspace{3cm}
	Signatures\par
	\begin{multicols}{2}
	    \begin{center}
	        Président \\
	        Secrétaire
	    \end{center}
	\end{multicols}
    
\end{document}
