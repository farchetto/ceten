%%%%%%%%%%%%%%%%%%%%%%%%%%%%%%%%%%%%%%%%%%%%%%%%%%%%%%%%%%%%%%%%%%%%%%%
%                                                                     %
%                        Charte des clubs                             %
%                                                                     %
%%%%%%%%%%%%%%%%%%%%%%%%%%%%%%%%%%%%%%%%%%%%%%%%%%%%%%%%%%%%%%%%%%%%%%%
%
%  Auteur :
%    Julien Déoux <julien.deoux@telecomnancy.net>
%    
%  Licence :
%    CC-BY-NC 4.0 (http://creativecommons.org/licenses/by-nc/4.0)
%
%%%



%%%%%%%%%%%%%%%%%%%
%  Configuration  %
%%%%%%%%%%%%%%%%%%%

\documentclass{article}
\usepackage[a4paper,includeheadfoot,margin=2.54cm]{geometry}
\usepackage[francais]{babel}

\usepackage{hyperref}
\usepackage{graphicx}
\usepackage{titlesec}
\usepackage[usenames]{xcolor}
\definecolor{newCeten}{RGB}{130,11,95}
\usepackage{enumitem,amssymb}
\newlist{todolist}{itemize}{2}
\setlist[todolist]{label=$\square$}
\usepackage{multicol}
\usepackage{setspace}
\usepackage{booktabs}
\usepackage{float}

\usepackage{fontspec} 
\setmainfont{Roboto Slab}
\setsansfont{Roboto}
\newfontfamily\condensed{Roboto Condensed}
\newfontfamily\condensedlight{Roboto Condensed Light}
\newfontfamily\light{Roboto Slab Light}
\titleformat*{\section}{\large\condensedlight\color{newCeten}}

\title{Charte des clubs}
\author{Julien Déoux}
\date\today

%%%%%%%%%%%%%%
%  Document  %
%%%%%%%%%%%%%%

\begin{document}

	%---------------%
	% Page de garde %
	%---------------%
	
	\begin{titlepage}
		\begin{center}
			\includegraphics[width=\textwidth]{images/ceten.png}\par
			\vspace{0.5cm}
			{\Huge \light{} Charte des clubs}\par
			\vspace{1cm}
		\end{center}
		\begin{center}
			club: \underline{\hspace{8cm}}
			année: \underline{\hspace{1.5cm}}\\
			\vspace{\baselineskip}
			$\square$ création du club
			\hspace{3cm}
			$\square$ renouvellement du bureau du club\\
			\vspace{\baselineskip}
			\begin{spacing}{1.5}
				président: \underline{\hspace{7cm}}\\
				trésorier: \underline{\hspace{7cm}}\\
				secrétaire: \underline{\hspace{7cm}}\\
				référent de TELECOM Nancy: \underline{\hspace{7cm}}\\
			\end{spacing}
		\end{center}
		type du club:
		\begin{todolist}
		\item club de services
		\item club de loisirs
		\item club évènementiel (Mois de passation:
			\underline{\hspace{5cm}})
		\end{todolist}
		\vspace{\baselineskip}
		objet du club:\\
		\hspace{\textwidth}\\
		\underline{\hspace{\textwidth}}\\
		\hspace{\textwidth}\\
		\underline{\hspace{\textwidth}}\\
		\hspace{\textwidth}\\
		\underline{\hspace{\textwidth}}\\

		\vfill
		\begin{center}
			{\footnotesize \light{} Ce texte régit le fonctionnement, les
			obligations et les droits attribués aux différents clubs composant
			le Ceten, comme définis dans le règlement intérieur. Un exemplaire
			est remis au président du club, l’autre est archivé au sein du
			BDE\@.\\
			Charte modifiée et votée le lundi 7 novembre 2016 par le BDE du
			Ceten}
		\end{center}
	\end{titlepage}

	%----------%
	% Articles %
	%----------%

	\pagenumbering{arabic}

	\begin{multicols}{2}
		
		\section{Définition}
\label{sec:definition}
			
		{\small
			
			Un «club» est une entité interne au Ceten, composée de membres et
			dédiée à un but ou une activité particulière. Le fonctionnement et
			les engagements pris par ce club sont détaillés dans cette Charte
			des clubs. L’objet spécifique du club est défini au début de cette
			charte.

			L'existence de ce club est subordonnée à la signature de la présente
			charte par le responsable des clubs du BDE, le référent de TELECOM
			Nancy et par les président, trésorier et secrétaire du club.

			Faisant partie intégrante de l’association, toutes les règles et
			décisions s’imposant au Ceten s’appliquent également au club.
			
		}

		\section{Type}
\label{sec:type}
			
		{\small
		
			Trois types de clubs sont définis au sein du Ceten:
			\begin{itemize}
				\item Les clubs de type évènementiel, dont le but est
					d’organiser ou de participer à une ou des manifestations
					ponctuelles
				\item Les clubs de type services, qui apportent un service
					particulier dans un domaine précis aux membres du Ceten
				\item Les clubs de type loisirs, regroupant les autres clubs et
					qui sont généralement axés sur le divertissement
			\end{itemize}
			Le type du club doit être défini avec le BDE lors de sa création et
			notifié au début de la charte.
			
		}

		\section{Composition}
\label{sec:composition}

		{\small
		
			Un club est composé de membres qui doivent également être membres
			adhérents du Ceten. Ce club se compose au minimum de trois personnes
			distinctes qui sont:
			\begin{itemize}
				\item le président, dont le rôle est défini en
					Section~\ref{sec:presidence}
				\item le trésorier, dont le rôle est défini en
					Section~\ref{sec:tresorerie}
				\item le secrétaire, dont le rôle est défini en
					Section~\ref{sec:secretariat}
			\end{itemize}

			Ces trois personnes déterminent le «bureau restreint» du club, qui a
			pour fonction de gérer l'ensemble du club. Le club n'a lieu
			d'exister sans la présence de ces six personnes.

			La structure interne du club et laissée libre au bureau restreint.
			En particulier, il peut désigner un «bureau complémentaire» dont le
			rôle est d'aider le bureau restreint à assurer ses fonctions, et
			dont la constitution doit être immédiatement transmise au
			responsable des clubs du BDE\@.

			L'ensemble de ces deux bureaux constitue le «bureau étendu» du club.
			Dans la suite de cette charte, l'appellation «bureau» désigne le
			bureau étendu du club.

			Le bureau du club doit tenir une liste à jour de ses membres et la
			laisser à disposition du BDE à tout moment.

		}

		\section{Durée}
\label{sec:duree}

		{\small
			
			La présente charte n'est valable que pendant la durée du mandat du
			bureau du club.

			Dans le cas d'un club évènementiel, le mandat du bureau commence le
			1\up{er} du mois suivant celui spécifié au début de la charte et
			s'achève un an plus tard, sauf en cas de réélection d'un ou
			plusieurs membres du bureau restreint du club.

			Dans les autres cas, un mandat du bureau commence au 1\up{er}
			janvier et se termine au 31 décembre de l'année spécifiée au début
			de la charte, sauf en cas de réélection d'un ou plusieurs membres du
			bureau restreint du club.

			Un membre du club l'est de la date de son inscription jusqu'au
			1\up{er} septembre suivant.

		}

		\section{Création}
\label{sec:creation}
		
		{\small
		
			La création d’un club se fait par trois membres du Ceten définissant
			leur rôle dans ledit club (président, trésorier et secrétaire). Le
			référent de TELECOM Nancy est désigné par la Direction des Études.
			Toute demande de création de club doit être soumise au vote du BDE\@.
			En cas d’accord et la charte des clubs signée, les fondateurs du
			club deviennent membres du club et définissent le bureau du club.
			
		}

		\section{Activité au sein du club}
\label{sec:activite_au_sein_du_club}
		
		{\small
		
			Le club se limite aux activités décrites par l’objet défini au début
			de la charte et uniquement par celui-ci. Il mettra en œuvre les
			moyens humains et logistiques pour parvenir au mieux à la
			réalisation de ces activités. Toute activité n’ayant pas de relation
			directe avec l’objet du club doit faire avant réalisation une
			demande auprès du BDE\@. Le BDE aidera, dans la mesure du possible et
			du raisonnable, le club à réaliser ses activités, par un 
			financement, du matériel, des formations et un suivi.

			Pour s’assurer que le club exerce correctement son activité, le BDE
			peut convoquer le club pour une de ses réunions hebdomadaires ou lui
			demander un compte-rendu d’activités à tout moment.
			
		}

		\section{Prise de décisions}
\label{sec:prise_de_decisions}
		
		{\small
		
			En période scolaire, chaque club doit se réunir à un minimum d’une
			fois tous les deux mois sur convocation du président ou à la demande
			du tiers de ses membres. Pour les clubs de type événementiel, deux
			mois avant l’évènement, cette période est réduite à une fois tous
			les quinze jours.

			Le BDE est invité permanent des réunions du bureau. La présence de
			membres du club ou de personnes extérieures au club est laissée à la
			discrétion du président. Le BDE se réserve le droit de refuser la
			présence en réunion d'une personne extérieure au Ceten.

			La prise de décisions se fait à la majorité absolue des voix
			exprimées. En cas d’égalité, la voix du président est prépondérante.
			Toutes les délibérations feront l’objet d’une retranscription écrite
			mise à disposition des membres de l’association. Si une délibération
			est contestée par la majorité absolue des membres du club, celle-ci
			est annulée et doit être reconsidérée à la prochaine réunion du
			club.

			Chaque réunion fera l’objet d’un compte-rendu qui sera archivé et
			transmis à l'ensemble du bureau du club ainsi qu'au BDE dans les
			cinq jours suivant la réunion.
			
		}

		\section{Propriété du club}
\label{sec:propriete_du_club}

		{\small

			Aucun bien ne peut appartenir au club. Cependant, des biens sont
			confiés par le BDE au club en fonction de ses besoins. Le président
			du club est responsable des biens prêtés par le BDE\@. La liste des
			biens prêtés devra être établie entre le club et le Responsable
			logistique du BDE, et être mise à jour régulièrement.

			Le BDE doit, dans la mesure des fonds et matériels disponibles,
			fournir au club l’ensemble des moyens nécessaires à la réalisation
			de ses activités. Ces biens peuvent être utilisés par d’autres
			membres du Ceten que ceux composant le club, cependant cette
			utilisation doit être soumise à l’approbation du président du club.

			Le BDE se réserve, à titre exceptionnel, le droit d’utiliser un des
			biens confié au club en prévenant son président. Un club peut
			également se voir attribuer, sous décision du BDE, des locaux à
			accès exclusif ou partagés avec d’autres clubs. Cet accès est
			réservé aux bureaux des clubs concernés et au BDE\@. L’accès à ces
			locaux aux autres membres du club est laissé à la discrétion des
			bureaux des clubs concernés et au BDE\@.

		}

		\section{Présidence}
\label{sec:presidence}

		{\small

			Le président est le représentant du club auprès du BDE\@. Avec le
			trésorier et le secrétaire, il est responsable des biens attribués
			au club, de ses membres et de toutes les décisions prises au nom de
			celui-ci. À ce titre, il s'assure auprès du responsable des clubs
			que le suivi du club par le BDE est suffisant. 

		}

		\section{Trésorerie}
\label{sec:tresorerie}

		{\small
		
			L’état des dépenses et recettes du club doit être géré par le
			trésorier du club. Il tient informé régulièrement le trésorier du
			BDE des flux d’argent au sein du club. L’utilisation d’un échéancier
			est encouragé pour savoir quelles recettes et dépenses ont été
			réellement faites. Chaque mouvement doit être justifié par une pièce
			comptable (factures, etc.). Toutes ces pièces comptables devront
			être archivées au sein du local du club, ou au BDE si le club ne
			possède pas de local. De plus, une copie numérique de ces pièces
			doit être transmise au BDE\@.

			Le président et le trésorier sont tenus responsables de la
			comptabilité du club. Ils doivent veiller au respect des budgets
			attribués par le BDE au début de l’année civile et doivent présenter
			un budget final, ainsi qu’un budget prévisionnel pour leur mandat ou
			le mandat suivant (selon mois de passation). Il convient de préciser
			les subventions demandées au BDE\@. Ces documents seront envoyés
			lors du mois de décembre au trésorier du BDE pour le bilan annuel et
			la création du budget prévisionnel du CETEN\@.

			Le budget prévisionnel de chaque club doit être nécessairement
			équilibré, sauf accord du trésorier du BDE\@.

			Toute transaction nécessitant l’utilisation du compte bancaire du
			BDE (paiement sur internet, émission ou encaissement d’un chèque,
			etc.) ne peut être faite que par le trésorier du BDE\@ ou avec
			l'accord de celui-ci, et sous les conditions qu'il estime
			nécessaires le cas échéant. Tout manquement à ces conditions ou abus
			de la part d'un membre du club pourra faire l'objet d'une sanction
			prononcée par le BDE\@.

			Un membre du club peut fournir une avance pour un achat du club. Le
			remboursement est validé et effectué par la trésorerie du BDE sous
			présentation d’une facture justifiant la dépense occasionnée et par
			la signature de la feuille de remboursement faites par le BDE\@. Le
			trésorier du club doit informer le trésorier du BDE pour tout achat
			que le club doit effectuer. La trésorerie du BDE et le président du
			BDE se réservent le droit de refuser le remboursement un achat si
			celui-ci n’est pas justifié. Toutes ces pièces comptables devront
			être données à la trésorerie du BDE pour archivage.

			Chaque club peut, s’il le souhaite, posséder une caisse
			monnaie. Le trésorier doit tenir à jour un journal de caisse
			enregistrant les entrées/sorties d’argent de la caisse du
			club. Ce journal de caisse devra être remis en fin d’année au
			trésorier du BDE\@.

		}

		\section{Secrétariat}
\label{sec:secretariat}

		{\small
		
			À chaque réunion du club, le secrétaire tient le compte-rendu et le
			transmet au reste du bureau et au BDE selon les modalités définies
			dans l'article~\ref{sec:prise_de_decisions}. Il est également, avec
			le président, responsable de la gestion des membres du club.
		
		}

		\section{Représentation du Ceten}
\label{sec:representation_du_ceten}

		{\small

			Le club n’est en aucun cas une entité morale. Il représente
			uniquement le Ceten pour l’activité décrite en objet de la charte et
			ne peut s’octroyer aucune responsabilité de l’association.

			Toute demande de sponsors, devis, accord, et autre contrat écrit
			entre le club et une personne morale doit être explicitement faite
			au nom du Ceten. Seules les personnes compétentes du BDE possèdent
			les droits de signature. La responsabilité est alors portée sur
			l’association.

		}

		\section{Élection du nouveau bureau}
\label{sec:election_du_nouveau_bureau}

		{\small

			Tout club est responsable du renouvellement du bureau et de son
			élection. Peuvent se présenter au bureau du club uniquement les
			membres qui ont adhéré au club au moins 30 jours avant le scrutin
			pour les clubs de services ou loisirs, et tous les membres Ceten
			pour les clubs de type événementiel. Chaque élection du bureau doit
			se faire en présence d’au moins un représentant du BDE\@.

			Le vote se fait à main levée. Un vote à bulletin secret peut être
			admis si la majorité des membres présents ou le Bureau du club le
			souhaite. Le vote par procuration est admis dans la limite d’une
			procuration par personne.

			La passation doit se faire durant le dernier mois du mandat du club
			ou, pour les clubs de type évènementiel, dans le mois de passation
			spécifié au début de la charte.

			Pour les clubs de type services et évènementiel, le scrutin est
			ouvert à tous les membres du Ceten. Pour les clubs de type loisirs,
			le scrutin est réservé aux uniques membres du club, qui ont dû
			adhérer à celui-ci au moins 30 jours avant l’élection.

			L’élection se fait par poste (président, trésorier, secrétaire) en
			deux tours. Pour chaque poste est élue au premier tour la personne
			ayant obtenu la majorité absolue des suffrages. Si aucune personne
			ne remporte la majorité absolue des voix lors du premier tour, un
			second tour est organisé à la majorité relative avec les deux
			personnes ayant reçu le plus de voix au premier tour. En cas
			d’égalité des voix, c’est au président sortant du club de
			départager.

			Le nouveau bureau prend ses fonctions, après signature de la charte,
			le 1er Janvier de l’année civile suivante. Pour les clubs de type
			événementiel, la date de prise de fonction est 	définie à la date de
			signature de la charte par le nouveau bureau.

			L’élection du nouveau bureau doit faire, comme pour toute réunion du
			club, l’objet d’un compte-rendu.

		}

		\section{Passation du club}
\label{sec:passation_du_club}

		{\small

			Durant le mois suivant l’élection, hors vacances scolaires, l’ancien
			bureau du club doit assister le nouveau bureau pour sa prise de
			fonctions future.

			Un dossier de passation écrit doit être présent dans tous les clubs.
			Celui-ci contiendra toutes les informations nécessaires pour la
			bonne continuité du club. Il devra être enrichi au fur et à mesure
			par les mandats des différents bureaux.

			De même, le bureau sortant peut désigner en accord avec le nouveau
			un responsable passation qui sera l’interlocuteur privilégié du
			nouveau bureau avec l’ancien. Son rôle est d’assister le nouveau
			bureau dans sa prise de fonctions.

		}

		\section{Référent Alisé}
\label{sec:referent_alise}

		{\small

			Pour certains clubs à grande importance au sein du Ceten,
			l’association Alisé peut mettre à disposition du club, en accord
			avec le président du club ou avec le BDE, un ou plusieurs référents.
			Ayant déjà eu une expérience au sein de ce club, ils peuvent
			apporter des conseils et éviter de reproduire les erreurs commises
			dans le passé. Le club est tenu d’échanger régulièrement avec le
			référent Alisé sur son activité. Tous les comptes-rendus des
			réunions devront également être envoyés au référent Alisé\@.

		}
		
		\section{Points CIPA et référent de TELECOM Nancy}
\label{sec:points_cipa_et_referent_de_telecom_nancy}

		{\small

			La participation à un club peut donner accès à un certain nombre de
			points CIPA pour l’obtention du diplôme de TELECOM Nancy. Il en
			revient à la direction de TELECOM Nancy de définir, dans le
			fascicule 0 du livret de l'élève, le nombre de points maximum
			attribués à la participation du club en fonction de la catégorie du
			club, fixée par le responsable des clubs et la direction des études,
			et de la fonction prise par les membres.

			Le tableau~\ref{fig:cipa} récapitule les maxima de points CIPA
			disponibles pour chaque catégorie. Il reprend les maxima définis
			dans le fascicule 0. En cas de différence, ce dernier fait foi.

			\begin{figure}[H]
				\centering
				\begin{tabular*}{\columnwidth}{@{} r @{\extracolsep{\fill}} *{3}{c} @{}}
					\toprule
					& Président & Bureau & Membres\\
					\midrule
					Catégorie 1 & 12 & 9 & 5\\
					\midrule
					Catégorie 2 & 9 & 7 & 4\\
					\midrule
					Catégorie 3 & 7 & 5 & 4\\
					\midrule
					Catégorie 4 & 5 & 3 & 1\\
					\bottomrule
				\end{tabular*}
				\caption{Récapitulatif des maxima de CIPA}
\label{fig:cipa}
			\end{figure}

			Pour se voir attribuer les points CIPA, le club possède un référent
			pouvant être un enseignant ou un membre de l’équipe d’administration
			de TELECOM Nancy. Ce référent est désigné par la direction des
			études de TELECOM Nancy au début de chaque mandat du club et a pour
			rôle de conseiller le bureau dans la gestion de son club dans ses
			activités. Il est l’interlocuteur privilégié pour les questions de
			fonctionnement liées à l’administration. Il doit signer cette
			présente charte pour que l’attribution des points CIPA puisse être
			validée. Un membre enseignant ou administratif de TELECOM Nancy peut
			être référent d’un ou plusieurs clubs. Pour permettre au référent de
			suivre son activité, le club devra lui envoyer un compte-rendu
			d’activité au minimum deux fois par an.

			Dans le dernier mois du mandat du club, le président du club doit
			soumettre au référent, au directeur des études ainsi qu’au BDE un
			compte-rendu de l’année du club comprenant:

			\begin{itemize}
				\item Un bilan moral
				\item Un bilan financier
				\item La liste des membres actifs du club en y précisant leur
					fonction et le travail réalisé par ceux-ci lors de l’année.
					De plus, le président attribue à chaque membre un nombre de
					points CIPA compris entre zéro et le maximum autorisé par le
					fascicule 0 selon son implication. Il y soumet également les
					points qu'il souhaiterait se voir attribuer. Le BDE et le
					référent de TELECOM Nancy peuvent demander une révision du
					compte-rendu et du nombre de points CIPA s’ils jugent
					ceux-ci non conformes à l’activité du club.
			\end{itemize}

		}
		
		\section{Perte de la qualité de membre du club}

		{\small

			La qualité de membre du club se perd par démission, perte
			de la qualité de membre du Ceten ou exclusion prononcée
			par le bureau du club ou le BDE\@. Une demande d’exclusion
			d’un membre peut être demandée au BDE par les deux-tiers
			des membres du club.

			Le bureau du club ou le BDE peut prononcer l’exclusion d’un
			membre du club pour motif grave. Le membre concerné
			sera convoqué par ledit bureau pour plaider sa cause. La
			décision devra être approuvée par les deux tiers du bureau.
			Le vote par procuration n’est pas permis.

			Si ce membre faisait partie du Bureau, le BDE devra mettre
			en place au sein du club une élection pour désigner son
			remplaçant. Une fois l’élection faite, la Charte devra être
			signée par le nouveau bureau. Si ce poste ne peut être
			pourvu par une autre personne, le club est dissout.

		}
		
	\end{multicols}

	\vfill
	Fait en deux exemplaires à Villers-lès-Nancy le \underline{\hspace{5cm}}
	\vfill

	\begin{multicols}{2}
		Le président du club \\
		Le secrétaire du club 
	\end{multicols}
		\vspace*{4cm}
	\begin{multicols}{2}
		Le trésorier du club \\
		Le référent de TELECOM Nancy 
	\end{multicols}
		\vspace*{4cm}
	\begin{multicols}{2}
		Le président du BDE \\
		Le responsable des clubs du BDE 
	\end{multicols}
		\vspace*{4cm}

\end{document}
